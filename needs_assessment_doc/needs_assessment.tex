%%%%%%%%%%%%%%%%%%%%%%%%%%%%%%%%%%%%%%%%%
% Arsclassica Article
% LaTeX Template
% Version 1.1 (10/6/14)
%
% This template has been downloaded from:
% http://www.LaTeXTemplates.com
%
% Original author:
% Lorenzo Pantieri (http://www.lorenzopantieri.net) with extensive modifications by:
% Vel (vel@latextemplates.com)
%
% License:
% CC BY-NC-SA 3.0 (http://creativecommons.org/licenses/by-nc-sa/3.0/)
%
%%%%%%%%%%%%%%%%%%%%%%%%%%%%%%%%%%%%%%%%%

%----------------------------------------------------------------------------------------
%	PACKAGES AND OTHER DOCUMENT CONFIGURATIONS
%----------------------------------------------------------------------------------------

\documentclass[
10pt, % Main document font size
a4paper, % Paper type, use 'letterpaper' for US Letter paper
oneside, % One page layout (no page indentation)
%twoside, % Two page layout (page indentation for binding and different headers)
headinclude,footinclude, % Extra spacing for the header and footer
BCOR5mm, % Binding correction
]{scrartcl}

%%%%%%%%%%%%%%%%%%%%%%%%%%%%%%%%%%%%%%%%%
% Arsclassica Article
% Structure Specification File
%
% This file has been downloaded from:
% http://www.LaTeXTemplates.com
%
% Original author:
% Lorenzo Pantieri (http://www.lorenzopantieri.net) with extensive modifications by:
% Vel (vel@latextemplates.com)
%
% License:
% CC BY-NC-SA 3.0 (http://creativecommons.org/licenses/by-nc-sa/3.0/)
%
%%%%%%%%%%%%%%%%%%%%%%%%%%%%%%%%%%%%%%%%%

%----------------------------------------------------------------------------------------
%	REQUIRED PACKAGES
%----------------------------------------------------------------------------------------

\usepackage[
nochapters, % Turn off chapters since this is an article        
beramono, % Use the Bera Mono font for monospaced text (\texttt)
eulermath,% Use the Euler font for mathematics
pdfspacing, % Makes use of pdftex’ letter spacing capabilities via the microtype package
dottedtoc % Dotted lines leading to the page numbers in the table of contents
]{classicthesis} % The layout is based on the Classic Thesis style

\usepackage{arsclassica} % Modifies the Classic Thesis package

\usepackage[T1]{fontenc} % Use 8-bit encoding that has 256 glyphs

\usepackage[utf8]{inputenc} % Required for including letters with accents

\usepackage{graphicx} % Required for including images
\graphicspath{{Figures/}} % Set the default folder for images

\usepackage{enumitem} % Required for manipulating the whitespace between and within lists

\usepackage{lipsum} % Used for inserting dummy 'Lorem ipsum' text into the template

\usepackage{subfig} % Required for creating figures with multiple parts (subfigures)

\usepackage{amsmath,amssymb,amsthm} % For including math equations, theorems, symbols, etc

\usepackage{varioref} % More descriptive referencing

%----------------------------------------------------------------------------------------
%	THEOREM STYLES
%---------------------------------------------------------------------------------------

\theoremstyle{definition} % Define theorem styles here based on the definition style (used for definitions and examples)
\newtheorem{definition}{Definition}

\theoremstyle{plain} % Define theorem styles here based on the plain style (used for theorems, lemmas, propositions)
\newtheorem{theorem}{Theorem}

\theoremstyle{remark} % Define theorem styles here based on the remark style (used for remarks and notes)

%----------------------------------------------------------------------------------------
%	HYPERLINKS
%---------------------------------------------------------------------------------------

\hypersetup{
%draft, % Uncomment to remove all links (useful for printing in black and white)
colorlinks=true, breaklinks=true, bookmarks=true,bookmarksnumbered,
urlcolor=webbrown, linkcolor=RoyalBlue, citecolor=webgreen, % Link colors
pdftitle={}, % PDF title
pdfauthor={\textcopyright}, % PDF Author
pdfsubject={}, % PDF Subject
pdfkeywords={}, % PDF Keywords
pdfcreator={pdfLaTeX}, % PDF Creator
pdfproducer={LaTeX with hyperref and ClassicThesis} % PDF producer
} % Include the structure.tex file which specified the document structure and layout

\definecolor{mygreen}{rgb}{0,0.6,0}
\definecolor{mygray}{rgb}{0.5,0.5,0.5}
\definecolor{mymauve}{rgb}{0.58,0,0.82}

\usepackage{natbib}
\usepackage{listings}

\lstset{ %
  backgroundcolor=\color{white},   % choose the background color; you must add \usepackage{color} or \usepackage{xcolor}
  basicstyle=\footnotesize\ttfamily,% the size of the fonts that are used for the code
  breakatwhitespace=false,         % sets if automatic breaks should only happen at whitespace
  breaklines=true,                 % sets automatic line breaking
  captionpos=t,                    % sets the caption-position to bottom
  commentstyle=\color{mygreen},    % comment style
  escapeinside={\%*}{*)},          % if you want to add LaTeX within your code
  extendedchars=true,              % lets you use non-ASCII characters; for 8-bits encodings only, does not work with UTF-8
  frame=single,                    % adds a frame around the code
  keepspaces=true,                 % keeps spaces in text, useful for keeping indentation of code (possibly needs columns=flexible)
  keywordstyle=\color{blue},       % keyword style
  language=C,                 % the language of the code
  otherkeywords={uint, size_t, ubyte, inline, __global__},
  numbers=left,                    % where to put the line-numbers; possible values are (none, left, right)
  numbersep=5pt,                   % how far the line-numbers are from the code
  numberstyle=\tiny\color{mygray}, % the style that is used for the line-numbers
  rulecolor=\color{black},         % if not set, the frame-color may be changed on line-breaks within not-black text (e.g. comments (green here))
  showspaces=false,                % show spaces everywhere adding particular underscores; it overrides 'showstringspaces'
  showstringspaces=false,          % underline spaces within strings only
  showtabs=false,                  % show tabs within strings adding particular underscores
  stepnumber=1,                    % the step between two line-numbers. If it's 1, each line will be numbered
  stringstyle=\color{mymauve},     % string literal style
  tabsize=2,                     % sets default tabsize to 2 spaces
}

\hyphenation{Fortran hy-phen-ation} % Specify custom hyphenation points in words with dashes where you would like hyphenation to occur, or alternatively, don't put any dashes in a word to stop hyphenation altogether

%----------------------------------------------------------------------------------------
%	TITLE AND AUTHOR(S)
%----------------------------------------------------------------------------------------

\title{\normalfont\spacedallcaps{Needs assessment for BioMap Mobile Application}} % The article title

\author{\spacedlowsmallcaps{Graeme Faul\textsuperscript{1} \& Gregory Linklater\textsuperscript{1}}} % The article author(s) - author affiliations need to be specified in the AUTHOR AFFILIATIONS block

\date{\today} % An optional date to appear under the author(s)

%----------------------------------------------------------------------------------------

\begin{document}

%----------------------------------------------------------------------------------------
%	HEADERS
%----------------------------------------------------------------------------------------

\renewcommand{\sectionmark}[1]{\markright{\spacedlowsmallcaps{#1}}} % The header for all pages (oneside) or for even pages (twoside)
%\renewcommand{\subsectionmark}[1]{\markright{\thesubsection~#1}} % Uncomment when using the twoside option - this modifies the header on odd pages
\lehead{\mbox{\llap{\small\thepage\kern1em\color{halfgray} \vline}\color{halfgray}\hspace{0.5em}\rightmark\hfil}} % The header style

\pagestyle{scrheadings} % Enable the headers specified in this block

%----------------------------------------------------------------------------------------
%	TABLE OF CONTENTS & LISTS OF FIGURES AND TABLES
%----------------------------------------------------------------------------------------

\maketitle % Print the title/author/date block

\setcounter{tocdepth}{2} % Set the depth of the table of contents to show sections and subsections only

\tableofcontents % Print the table of contents

% \listoffigures % Print the list of figures

% \listoftables % Print the list of tables



%----------------------------------------------------------------------------------------
%	AUTHOR AFFILIATIONS
%----------------------------------------------------------------------------------------

{\let\thefootnote\relax\footnotetext{\textsuperscript{1} \textit{Department of Computer Science, Rhodes University, Grahamstown, South Africa}}}

%----------------------------------------------------------------------------------------

\newpage % Start the article content on the second page, remove this if you have a longer abstract that goes onto the second page

\section{Problem Specification} % (fold)
\label{sec:problem_specification}

\subsection{Background} % (fold)
\label{sub:background}

The Animal Demography Unit (ADU) is an online database of fauna and flora sightings from around the world. It is comprised of records sent in by various individuals from around the world. Each record notes the date, location, classification of the species of the subject of any particular entry. It also includes one or more pictures of the subject and the name of the individual who submitted the record. Each record undergoes a verification process to determine its validity before it is added to the database.

% subsection background (end)

\subsection{The Problem} % (fold)
\label{sub:the_problem}

The problem outlined by the client is that the system for recording new entries is slow and outdated. The current process for creating a new record is done manually on a web based system with the individual having to capture everything themselves. This includes location, pictures (which often have to be transferred from camera to a PC and then uploaded), personal details, date and species. This is a very time consuming process and also presents the possibility of incorrect data capture. From this, a clear need for an automated and modern solution can be identified.   

% subsection the_problem (end)

% section problem_specification (end)

\section{Existing System} % (fold)
\label{sec:existing_system}

Assuming the individual has a registered account on ADU, the steps for adding  a new record are as follows:

\begin{enumerate}
\item Log in to the website
\item Click ``Data upload'' link
\item Fill in all fields, namely: Additional observers, date (will default to current date), and location (exact latitude and longitude can be determined using Google Map widget)
\item Save the record and continue to next page
\end{enumerate}

On the second page, the user can enter up to three records at once on any of the various projects. For each record the user must:

\begin{enumerate}
\item Select the project for which they wish the record to be entered in
\item Enter the collection date (will default to current date)
\item Upload up to three pictures (Most likely copied from a camera enabled device)
\item Upload a sound file (for use on FrogMAP only)
\item Identify the species
\item Enter any additional notes
\item Enter nest count and type (for use on PHOWN only)
\end{enumerate}

% section existing_system (end)

\section{Needs of the Client} % (fold)
\label{sec:needs_of_the_client}

The client would like an android smart phone application to be developed that will reduce the human error in the process of capturing information of specimens as mentioned in Section~\vref{sub:the_problem}. In it's simplest form, users of the application should be able to submit photos and metadata of specimens found in the wild while automating as much of the data capture process as possible. This serves to decrease the barrier to entry for participation in this citizen science project and therefore would hopefully increase participation. \\

\noindent
The client has requested that the application be created with non-technical users in mind; meaning that the application should be extremely simple to operate, with very little interaction needed to get to the main purpose of the application -- namely taking pictures of newly found specimens. Following that, the user should be allowed to optionally add descriptions and additional media to the submission and should confirm the automatically entered data such as the date and locality of the specimen. Once the submission is made then the users' part of the process is done and they can move to new specimens to collect. 

\subsection{Client Requests} % (fold)
\label{sub:client_requests}

The specific requests of the client are as follows:
\begin{enumerate}
\item The user must be able to submit new records for any one of the databases provided by the ADU.
\item Verify the identity of the user.
\item Submissions should contain the following information:
\begin{itemize}
  \item Media relating to the specimen found (photographs, sound files, etc.)
  \item Location data automatically obtained from the user's device.
  \item Date and time automatically obtained from the user's device.
  \item Optional universal or database specific metadata and textual descriptions of location and specimen
\end{itemize}
\item Allow the user to overwrite any automatically obtained data (namely date, time and location).
\item Detect poor photographs and notify user as such.
\item Option to only upload records when the user is connected to Wi-Fi.
\item Verify that all new records are complete before submission.
\end{enumerate}

% subsection client_requests (end)

% section needs_of_the_client (end)

\section{Assessment of Needs} % (fold)
\label{sec:assessment_of_needs}

Regarding most of the requirements (or needs) put forward by the client; 

\noindent
Using the built-in hardware and sensors of a smart phone it is possible to automate the process of capturing GPS, date and time whenever a picture is taken. It is further possible to automate data entry by having the user log in and select the database they wish to contribute to prior to taking a photo; this would allow us to attach user data to the submission and automatically submit it to the correct database. \\

% section assessment_of_needs (end)

%----------------------------------------------------------------------------------------
%	BIBLIOGRAPHY
%----------------------------------------------------------------------------------------

% \renewcommand{\refname}{\spacedlowsmallcaps{References}} % For modifying the bibliography heading

% \bibliographystyle{unsrt}

% \bibliography{sample.bib} % The file containing the bibliography

%----------------------------------------------------------------------------------------

\end{document}